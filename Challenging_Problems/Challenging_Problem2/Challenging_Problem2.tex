\documentclass[journal,12pt,twocolumn]{IEEEtran}

\usepackage{setspace}
\usepackage{gensymb}
\singlespacing
\usepackage[cmex10]{amsmath}

\usepackage{amsthm}

\usepackage{mathrsfs}
\usepackage{txfonts}
\usepackage{stfloats}
\usepackage{bm}
\usepackage{cite}
\usepackage{cases}
\usepackage{subfig}

\usepackage{longtable}
\usepackage{multirow}

\usepackage{enumitem}
\usepackage{mathtools}
\usepackage{steinmetz}
\usepackage{tikz}
\usepackage{circuitikz}
\usepackage{verbatim}
\usepackage{tfrupee}
\usepackage[breaklinks=true]{hyperref}
\usepackage{graphicx}
\usepackage{tkz-euclide}

\usetikzlibrary{calc,math}
\usepackage{listings}
    \usepackage{color}                                            %%
    \usepackage{array}                                            %%
    \usepackage{longtable}                                        %%
    \usepackage{calc}                                             %%
    \usepackage{multirow}                                         %%
    \usepackage{hhline}                                           %%
    \usepackage{ifthen}                                           %%
    \usepackage{lscape}     
\usepackage{multicol}
\usepackage{chngcntr}

\DeclareMathOperator*{\Res}{Res}

\renewcommand\thesection{\arabic{section}}
\renewcommand\thesubsection{\thesection.\arabic{subsection}}
\renewcommand\thesubsubsection{\thesubsection.\arabic{subsubsection}}
\newcommand*{\Perm}[2]{{}^{#1}\!P_{#2}}%
\newcommand*{\Comb}[2]{{}^{#1}C_{#2}}%
\renewcommand\thesectiondis{\arabic{section}}
\renewcommand\thesubsectiondis{\thesectiondis.\arabic{subsection}}
\renewcommand\thesubsubsectiondis{\thesubsectiondis.\arabic{subsubsection}}
\renewcommand{\complement}[1]{\widetilde{#1}}   

\hyphenation{op-tical net-works semi-conduc-tor}
\def\inputGnumericTable{}                                 %%

\lstset{
%language=C,
frame=single, 
breaklines=true,
columns=fullflexible
}

\begin{document}
\newtheorem{theorem}{Theorem}[section]
\newtheorem{problem}{Problem}
\newtheorem{proposition}{Proposition}[section]
\newtheorem{lemma}{Lemma}[section]
\newtheorem{corollary}[theorem]{Corollary}
\newtheorem{example}{Example}[section]
\newtheorem{definition}[problem]{Definition}

\newcommand{\BEQA}{\begin{eqnarray}}
\newcommand{\EEQA}{\end{eqnarray}}
\newcommand{\define}{\stackrel{\triangle}{=}}
\bibliographystyle{IEEEtran}
\raggedbottom
\setlength{\parindent}{0pt}
\providecommand{\mbf}{\mathbf}
\providecommand{\pr}[1]{\ensuremath{\pr\left(#1\right)}}
\providecommand{\qfunc}[1]{\ensuremath{Q\left(#1\right)}}
\providecommand{\sbrak}[1]{\ensuremath{{}\left[#1\right]}}
\providecommand{\lsbrak}[1]{\ensuremath{{}\left[#1\right.}}
\providecommand{\rsbrak}[1]{\ensuremath{{}\left.#1\right]}}
\providecommand{\brak}[1]{\ensuremath{\left(#1\right)}}
\providecommand{\lbrak}[1]{\ensuremath{\left(#1\right.}}
\providecommand{\cbrak}[1]{\ensuremath{\left\{#1\right\}}}
\providecommand{\lcbrak}[1]{\ensuremath{\left\{#1\right.}}
\providecommand{\rcbrak}[1]{\ensuremath{\left.#1\right\}}}
\theoremstyle{remark}
\newtheorem{rem}{Remark}
\newcommand{\sgn}{\mathop{\mathrm{sgn}}}
% \providecommand{\abs}[1]{\left\vert#1\right\vert}
% \providecommand{\res}[1]{\Res\displaylimits_{#1}} 
% \providecommand{\norm}[1]{\left\lVert#1\right\rVert}

% \providecommand{\mtx}[1]{\mathbf{#1}}
% \providecommand{\mean}[1]{E\left[ #1 \right]}
% \providecommand{\fourier}{\overset{\mathcal{F}}{ \rightleftharpoons}}

\providecommand{\system}{\overset{\mathcal{H}}{ \longleftrightarrow}}

\newcommand{\solution}{\noindent \textbf{Solution: }}
\newcommand{\cosec}{\,\text{cosec}\,}
\providecommand{\dec}[2]{\ensuremath{\overset{#1}{\underset{#2}{\gtrless}}}}
\newcommand{\myvec}[1]{\ensuremath{\begin{pmatrix}#1\end{pmatrix}}}
\newcommand{\mydet}[1]{\ensuremath{\begin{vmatrix}#1\end{vmatrix}}}
\numberwithin{equation}{subsection}
\makeatletter
\@addtoreset{figure}{problem}
\makeatother
\let\StandardTheFigure\thefigure
\let\vec\mathbf
\renewcommand{\thefigure}{\theproblem}
\def\putbox#1#2#3{\makebox[0in][l]{\makebox[#1][l]{}\raisebox{\baselineskip}[0in][0in]{\raisebox{#2}[0in][0in]{#3}}}}
     \def\rightbox#1{\makebox[0in][r]{#1}}
     \def\centbox#1{\makebox[0in]{#1}}
     \def\topbox#1{\raisebox{-\baselineskip}[0in][0in]{#1}}
     \def\midbox#1{\raisebox{-0.5\baselineskip}[0in][0in]{#1}}
\vspace{3cm}
\title{AI1103: Challenging Problem 2}
\author{Chitneedi Geetha Sowmya \\ CS20BTECH11011}
\maketitle
\newpage
\bigskip
\renewcommand{\thefigure}{\theenumi}
\renewcommand{\thetable}{\theenumi}


Download all latex codes from 
\begin{lstlisting}
https://github.com/Geetha495/AI1103/Challenging_Problems/Challenging_Problem2/blob/main/Challenging_Problem2.tex
\end{lstlisting}


\section{Problem}
Suppose $X$ is a random variable such that $E(X)=0 , E(X^2)=2 , E(X^4)=4 $. Then 
\begin{enumerate}
    \item $E(X^3)=0$
    \item $\Pr(X\geq 0) = \frac{1}{2}$
    \item $X \sim N(0,2)$ 
\item $X$ is bounded with Probability 1.
\end{enumerate}

\section{Solution}
Let $Y=X^2$ be a random variable, Then 
\begin{align*}
    \sigma(Y) &= E(Y^2) - (E(Y))^{2} \\
    &= E(X^4) - (E(X^2))^2 \\
             &= 0
             \end{align*}
So, $Y$ is a constant random variable.\\
Thus, for all $x \in X$  , $x^2 = c $, where $c$ is constant.
\begin{enumerate}
    \item 
    \begin{align*}
        E(X^3) &= \sum_{x \in X} x^3 p_X(x) \\
        &= c \times \sum_{x \in X} x p_X(x)\\
        &= c \times E(X)\\
        &= 0
    \end{align*}
    Option 1 is correct.
    \item
    \begin{align*}
     E(X)  &= \sum_{x \in X} x p_X(x)  \\
       0 &=  \ \sqrt{c} \Pr(X \geq 0) + \left(-\sqrt{c} \Pr(X < 0 ) \right) \\
      \Pr(X \geq 0) &= \Pr(X < 0)
    \end{align*}
    \begin{align*}
        \text{As } \sum_{x \in X}  p_X(x) &= 1\\
      \Pr(X \geq 0) + \Pr(X < 0)  &= 1\\
      \Pr(X \geq 0) &= \frac{1}{2}
    \end{align*}
    Option 2 is correct.
    
    \item 
    \begin{align*}
    p_X(x) &= \begin{cases} 
      \frac{1}{2} & x = \pm \sqrt{c} \\
      0 & \text{otherwise} \\
   \end{cases} 
\end{align*} 
Hence $X$ forms a discrete probability distribution. So, it can't be normal distribution\\
Option 3 is wrong .\\

\item 
As $p_X(x)$ takes only values $\frac{1}{2}$ and 0 , $X$ is bounded.\\
Option 4 is correct.
\end{enumerate}
Thus, correct options are 1,2,4.
\end{document}




