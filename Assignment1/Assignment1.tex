\documentclass[journal,12pt,twocolumn]{IEEEtran}

\usepackage{setspace}
\usepackage{gensymb}
\singlespacing
\usepackage[cmex10]{amsmath}

\usepackage{amsthm}

\usepackage{mathrsfs}
\usepackage{txfonts}
\usepackage{stfloats}
\usepackage{bm}
\usepackage{cite}
\usepackage{cases}
\usepackage{subfig}

\usepackage{longtable}
\usepackage{multirow}

\usepackage{enumitem}
\usepackage{mathtools}
\usepackage{steinmetz}
\usepackage{tikz}
\usepackage{circuitikz}
\usepackage{verbatim}
\usepackage{tfrupee}
\usepackage[breaklinks=true]{hyperref}
\usepackage{graphicx}
\usepackage{tkz-euclide}

\usetikzlibrary{calc,math}
\usepackage{listings}
    \usepackage{color}                                            %%
    \usepackage{array}                                            %%
    \usepackage{longtable}                                        %%
    \usepackage{calc}                                             %%
    \usepackage{multirow}                                         %%
    \usepackage{hhline}                                           %%
    \usepackage{ifthen}                                           %%
    \usepackage{lscape}     
\usepackage{multicol}
\usepackage{chngcntr}

\DeclareMathOperator*{\Res}{Res}

\renewcommand\thesection{\arabic{section}}
\renewcommand\thesubsection{\thesection.\arabic{subsection}}
\renewcommand\thesubsubsection{\thesubsection.\arabic{subsubsection}}
\newcommand*{\Perm}[2]{{}^{#1}\!P_{#2}}%
\newcommand*{\Comb}[2]{{}^{#1}C_{#2}}%
\renewcommand\thesectiondis{\arabic{section}}
\renewcommand\thesubsectiondis{\thesectiondis.\arabic{subsection}}
\renewcommand\thesubsubsectiondis{\thesubsectiondis.\arabic{subsubsection}}


\hyphenation{op-tical net-works semi-conduc-tor}
\def\inputGnumericTable{}                                 %%

\lstset{
%language=C,
frame=single, 
breaklines=true,
columns=fullflexible
}
\begin{document}


\newtheorem{theorem}{Theorem}[section]
\newtheorem{problem}{Problem}
\newtheorem{proposition}{Proposition}[section]
\newtheorem{lemma}{Lemma}[section]
\newtheorem{corollary}[theorem]{Corollary}
\newtheorem{example}{Example}[section]
\newtheorem{definition}[problem]{Definition}

\newcommand{\BEQA}{\begin{eqnarray}}
\newcommand{\EEQA}{\end{eqnarray}}
\newcommand{\define}{\stackrel{\triangle}{=}}
\bibliographystyle{IEEEtran}
\raggedbottom
\setlength{\parindent}{0pt}
\providecommand{\mbf}{\mathbf}
\providecommand{\pr}[1]{\ensuremath{\pr\left(#1\right)}}
\providecommand{\qfunc}[1]{\ensuremath{Q\left(#1\right)}}
\providecommand{\sbrak}[1]{\ensuremath{{}\left[#1\right]}}
\providecommand{\lsbrak}[1]{\ensuremath{{}\left[#1\right.}}
\providecommand{\rsbrak}[1]{\ensuremath{{}\left.#1\right]}}
\providecommand{\brak}[1]{\ensuremath{\left(#1\right)}}
\providecommand{\lbrak}[1]{\ensuremath{\left(#1\right.}}
\providecommand{\rbrak}[1]{\ensuremath{\left.#1\right)}}
\providecommand{\cbrak}[1]{\ensuremath{\left\{#1\right\}}}
\providecommand{\lcbrak}[1]{\ensuremath{\left\{#1\right.}}
\providecommand{\rcbrak}[1]{\ensuremath{\left.#1\right\}}}
\theoremstyle{remark}
\newtheorem{rem}{Remark}
\newcommand{\sgn}{\mathop{\mathrm{sgn}}}
% \providecommand{\abs}[1]{\left\vert#1\right\vert}
% \providecommand{\res}[1]{\Res\displaylimits_{#1}} 
% \providecommand{\norm}[1]{\left\lVert#1\right\rVert}

% \providecommand{\mtx}[1]{\mathbf{#1}}
% \providecommand{\mean}[1]{E\left[ #1 \right]}
% \providecommand{\fourier}{\overset{\mathcal{F}}{ \rightleftharpoons}}

\providecommand{\system}{\overset{\mathcal{H}}{ \longleftrightarrow}}
	
\newcommand{\solution}{\noindent \textbf{Solution: }}
\newcommand{\cosec}{\,\text{cosec}\,}
\providecommand{\dec}[2]{\ensuremath{\overset{#1}{\underset{#2}{\gtrless}}}}
\newcommand{\myvec}[1]{\ensuremath{\begin{pmatrix}#1\end{pmatrix}}}
\newcommand{\mydet}[1]{\ensuremath{\begin{vmatrix}#1\end{vmatrix}}}
\numberwithin{equation}{subsection}
\makeatletter
\@addtoreset{figure}{problem}
\makeatother
\let\StandardTheFigure\thefigure
\let\vec\mathbf
\renewcommand{\thefigure}{\theproblem}
\def\putbox#1#2#3{\makebox[0in][l]{\makebox[#1][l]{}\raisebox{\baselineskip}[0in][0in]{\raisebox{#2}[0in][0in]{#3}}}}
     \def\rightbox#1{\makebox[0in][r]{#1}}
     \def\centbox#1{\makebox[0in]{#1}}
     \def\topbox#1{\raisebox{-\baselineskip}[0in][0in]{#1}}
     \def\midbox#1{\raisebox{-0.5\baselineskip}[0in][0in]{#1}}
\vspace{3cm}
\title{AI1103: Assignment 1}
\author{Chitneedi Geetha Sowmya \\ CS20BTECH11011}
\maketitle
\newpage
\bigskip
\renewcommand{\thefigure}{\theenumi}
\renewcommand{\thetable}{\theenumi}


Download all python codes from 
\begin{lstlisting}
https://github.com/Geetha495/AI1103/blob/main/Assignment1/Assignment1.py
\end{lstlisting}
%
and latex-tikz codes from 
%
\begin{lstlisting}
https://github.com/Geetha495/AI1103/blob/main/Assignment1/Assignment1.tex
\end{lstlisting}



\section{Problem}
Determine $\Pr( E|F)$, if a coin is tossed three
times
\begin{enumerate}[label=\roman*]
    \item E : head on third toss , F : heads on first
two tosses
\item E : at least two heads , F : at most two
heads
\item E : at most two tails , F : at least one tail
\end{enumerate}



\section{Solution}
In an experiment of tossing a coin $n$( = 3) times, random variable  $X \in \lbrace 0,1,2,3 \rbrace$ follows binomial distribution.\\
The binomial distribution formula is:
\begin{align*}
 \Pr( X=k ) &= \Comb{n}{k} \times p^k \times (1- p)^{n - k}
\end{align*}

Where:


\begin{table}[h]

    \centering
    \resizebox{\columnwidth}{!}{%
    \begin{tabular}{|r|c|}\hline
    $k$ &  total number of “successes” \\ \hline
    $p$ & probability of a success on an individual trial\\ \hline
    $n$ & number of trials = 3 \\ \hline
\end{tabular}}
\caption{The binomial distribution formula}
    \label{table:0}
\end{table}


\begin{enumerate}[label=(\roman*)]
    \item From table \ref{table:1}, $\Pr(E|F)$ = 0.5
    \item $X$ denotes number of heads. From table \ref{table:2}, $\Pr(E|F)$ = 0.428
    \item $X$ denotes number of tails. From table \ref{table:3}, $\Pr(E|F)$ = 0.857
\end{enumerate}


\begin{table}[ht]

    \centering
     \resizebox{\columnwidth}{!}{%
    \begin{tabular}{|r|l|} \hline
    $\Pr$(Event)  & Calculation   \\ \hline
    $\Pr( F)$    & From product rule , \\ 
    &= $\frac{1}{2}\times\frac{1}{2}$ \\ 
    &=  0.25 \\ \hline 
    $\Pr( EF)$   &  From product rule, \\   
    &= $\frac{1}{2}\times\frac{1}{2}\times\frac{1}{2}$ \\ 
    & = 0.125\\ \hline 
    $\Pr(E|F )$  &= $\frac{\Pr(EF)}{\Pr(F)} $ \\ 
    &  = 0.5 \\ \hline 
    \end{tabular} }
    \caption{Part(i)}
    \label{table:1}
\end{table}

\begin{table}[ht]

    \centering
    \resizebox{\columnwidth}{!}{%
     \begin{tabular}{|r|l|}\hline
      $\Pr$(Event)  & Calculation \\ \hline
      $\Pr( F)$ 
      &= $\Pr( X\leq2)$ \\ 
      &= $ \Pr( X=0) + \Pr( X=1) + \Pr( X=2 )$\\ 
      &= $\Comb{3}{0} \left(\frac{1}{2}\right)^3  + \Comb{3}{1} \left(\frac{1}{2}\right)^3 + \Comb{3}{2} \left(\frac{1}{2}\right)^3$\\ 
      &= 0.875 \\ \hline
    $\Pr( EF)$  &= $\Pr( X=2) $ \\
    &= 0.375 \\ \hline
    $\Pr( E|F )$   &=$ \frac{\Pr(EF)}{\Pr(F)} $ \\ 
    &= 0.428 \\ \hline
    \end{tabular}}
    \caption{Part(ii)}
    \label{table:2}
\end{table}


\begin{table}[ht]

       \centering
       \resizebox{\columnwidth}{!}{%
       \begin{tabular}{|r|l|}\hline
      $\Pr$(Event)  & Calculation \\ \hline
       $\Pr( F)$ &= $\Pr( X\geq1)$\\
        &= $1-\Pr( 0)$ \\
        &= 0.875 \\ \hline
        $\Pr( EF)$ &= $\Pr(X= 1) + \Pr(X= 2 )$ \\
         &= 0.75 \\ \hline
         $\Pr( E|F)$ &= $\frac{\Pr(EF)}{\Pr(F)} $ \\
        &= 0.857 \\ \hline
      \end{tabular} }
       \caption{Part(iii)}
       \label{table:3}
   \end{table}
  
  
     
\end{document}




