\documentclass[journal,12pt,twocolumn]{IEEEtran}

\usepackage{setspace}
\usepackage{gensymb}
\singlespacing
\usepackage[cmex10]{amsmath}

\usepackage{amsthm}

\usepackage{mathrsfs}
\usepackage{txfonts}
\usepackage{stfloats}
\usepackage{bm}
\usepackage{cite}
\usepackage{cases}
\usepackage{subfig}

\usepackage{longtable}
\usepackage{multirow}

\usepackage{enumitem}
\usepackage{mathtools}
\usepackage{steinmetz}
\usepackage{tikz}
\usepackage{circuitikz}
\usepackage{verbatim}
\usepackage{tfrupee}
\usepackage[breaklinks=true]{hyperref}
\usepackage{graphicx}
\usepackage{tkz-euclide}
\usepackage{cleveref}

\usetikzlibrary{calc,math}
\usepackage{listings}
    \usepackage{color}                                            %%
    \usepackage{array}                                            %%
    \usepackage{longtable}                                        %%
    \usepackage{calc}                                             %%
    \usepackage{multirow}                                         %%
    \usepackage{hhline}                                           %%
    \usepackage{ifthen}                                           %%
    \usepackage{lscape}     
\usepackage{multicol}
\usepackage{chngcntr}

\DeclareMathOperator*{\Res}{Res}

\renewcommand\thesection{\arabic{section}}
\renewcommand\thesubsection{\thesection.\arabic{subsection}}
\renewcommand\thesubsubsection{\thesubsection.\arabic{subsubsection}}

\renewcommand\thesectiondis{\arabic{section}}
\renewcommand\thesubsectiondis{\thesectiondis.\arabic{subsection}}
\renewcommand\thesubsubsectiondis{\thesubsectiondis.\arabic{subsubsection}}


\hyphenation{op-tical net-works semi-conduc-tor}
\def\inputGnumericTable{}                                 %%

\lstset{
%language=C,
frame=single, 
breaklines=true,
columns=fullflexible
}
\begin{document}


\newtheorem{theorem}{Theorem}[section]
\newtheorem{problem}{Problem}
\newtheorem{proposition}{Proposition}[section]
\newtheorem{lemma}{Lemma}[section]
\newtheorem{corollary}[theorem]{Corollary}
\newtheorem{example}{Example}[section]
\newtheorem{definition}[problem]{Definition}

\newcommand{\BEQA}{\begin{eqnarray}}
\newcommand{\EEQA}{\end{eqnarray}}
\newcommand{\define}{\stackrel{\triangle}{=}}
\bibliographystyle{IEEEtran}
\raggedbottom
\setlength{\parindent}{0pt}
\providecommand{\mbf}{\mathbf}
\providecommand{\pr}[1]{\ensuremath{\Pr\left(#1\right)}}
\providecommand{\qfunc}[1]{\ensuremath{Q\left(#1\right)}}
\providecommand{\sbrak}[1]{\ensuremath{{}\left[#1\right]}}
\providecommand{\lsbrak}[1]{\ensuremath{{}\left[#1\right.}}
\providecommand{\rsbrak}[1]{\ensuremath{{}\left.#1\right]}}
\providecommand{\brak}[1]{\ensuremath{\left(#1\right)}}
\providecommand{\lbrak}[1]{\ensuremath{\left(#1\right.}}
\providecommand{\rbrak}[1]{\ensuremath{\left.#1\right)}}
\providecommand{\cbrak}[1]{\ensuremath{\left\{#1\right\}}}
\providecommand{\lcbrak}[1]{\ensuremath{\left\{#1\right.}}
\providecommand{\rcbrak}[1]{\ensuremath{\left.#1\right\}}}
\theoremstyle{remark}
\newtheorem{rem}{Remark}
\newcommand{\sgn}{\mathop{\mathrm{sgn}}}
\providecommand{\abs}[1]{\ensuremath{\left \vert #1\right\vert}}
\providecommand{\res}[1]{\Res\displaylimits_{#1}} 
%\providecommand{\norm}[1]{\left\lVert#1\right\rVert}
%\providecommand{\norm}[1]{\lVert#1\rVert}
\providecommand{\mtx}[1]{\mathbf{#1}}
%\providecommand{\mean}[1]{E\left[ #1 \right]}
\providecommand{\fourier}{\overset{\mathcal{F}}{ \rightleftharpoons}}
%\providecommand{\hilbert}{\overset{\mathcal{H}}{ \rightleftharpoons}}
\providecommand{\system}{\overset{\mathcal{H}}{ \longleftrightarrow}}
	%\newcommand{\solution}[2]{\textbf{Solution:}{#1}}
\newcommand{\solution}{\noindent \textbf{Solution: }}
\newcommand{\cosec}{\,\text{cosec}\,}
\providecommand{\dec}[2]{\ensuremath{\overset{#1}{\underset{#2}{\gtrless}}}}
\newcommand*{\Comb}[2]{{}^{#1}C_{#2}}%
\newcommand{\myvec}[1]{\ensuremath{\begin{pmatrix}#1\end{pmatrix}}}
\newcommand{\mydet}[1]{\ensuremath{\begin{vmatrix}#1\end{vmatrix}}}
\numberwithin{equation}{subsection}
\makeatletter
\@addtoreset{figure}{problem}
\makeatother
\let\StandardTheFigure\thefigure
\let\vec\mathbf
\renewcommand{\thefigure}{\theproblem}
\def\putbox#1#2#3{\makebox[0in][l]{\makebox[#1][l]{}\raisebox{\baselineskip}[0in][0in]{\raisebox{#2}[0in][0in]{#3}}}}
     \def\rightbox#1{\makebox[0in][r]{#1}}
     \def\centbox#1{\makebox[0in]{#1}}
     \def\topbox#1{\raisebox{-\baselineskip}[0in][0in]{#1}}
     \def\midbox#1{\raisebox{-0.5\baselineskip}[0in][0in]{#1}}
\vspace{3cm}
\title{AI1103 Challenging Problem 20}
\author{Chitneedi Geetha Sowmya \\ CS20BTECH11011}
\maketitle
\newpage
\bigskip
\renewcommand{\thefigure}{\theenumi}
\renewcommand{\thetable}{\theenumi}
\newcommand{\dsum}{\displaystyle\sum}
\newcommand{\R}{\mathbb{R}}
\newcommand{\C}{\mathbb{C}}
% the python code from 
% \begin{lstlisting}
% https://github.com/Geetha495/AI1103/blob/main/Challenging_Problems/Challenging_Problem20/Challenging_Problem20.py\end{lstlisting}
% %
% and
Download  latex-tikz code from 
%
\begin{lstlisting}
https://github.com/Geetha495/AI1103/blob/main/Challenging_Problems/Challenging_Problem20/Challenging_Problem20.tex\end{lstlisting}

\section{Question}
Let $X_{1},X_{2},....$ be i.i.d N(1,1) random variables.Let $S_{n}=X_{1}^{2}+X_{2}^2+...+X_{n}^{2}$ for $n\ge1$.Then $$\lim_{n \to \infty}{\frac{Var\brak{S_{n}}}{n}}=$$
\begin{enumerate}[label = (\Alph*)]
\item  $4$
\item  $6$
\item  $1$
\item  $0$
\end{enumerate}
\section{Solution}
\begin{definition}[Central moment]
 For a random variable $X$, $E\left[(X-E(X))^r\right]$ is called $r^{th} $ central moment and it is denoted by $\mu_r$.
\begin{align}
    \mu_r &= E\brak{(X-E(X))^r} \nonumber\\
    &= \sum_{k=0}^r \brak{\Comb{4}{k} \times E(X^k) \times (E(X))^{r-k}}
    \label{eq:2}
\end{align}
\end{definition}
\begin{definition}[Kurtosis]
 It is the measure of tailedness of the probability distribution of a random variable $X$.
\begin{align*}
    \text{Kurtosis} &= \frac{\mu_4}{( Var(X) )^2}\\
\end{align*}
\end{definition}
 
As $X_{1},X_{2},...X_n$ are independently and identically distributed random variables, \begin{align*}
    E(X_{1})&=E(X_{2})=...=E(X_n) \\
    Var(X_{1})&=Var(X_{2})=...=Var(X_n)
\end{align*}
And from N(1,1), it is clear that for all
$1 \leq i \leq n$
\begin{align*}
    E(X_{i}) &=1 \\
    Var(X_{i})&= 1  
\end{align*}
So,
\begin{align*}
    E(X_{i}^2)  &=  (  Var(X_i) + (E({X_i}))^2 )\\
    &=1+1 = 2
\end{align*}
Given,\begin{align*}
    S_{n} &= X_{1}^{2}+X_{2}^2+...+X_{n}^{2}\\
    &= \sum_{i=1}^n{X_{i}^2} 
    \end{align*}
       \begin{align}
    Var(S_n) &= Var\left( \sum_{i=1}^n{X_{i}^2}\right)\nonumber  \\
   &= \sum_{i=1}^n{ Var(X_{i}^2)}\nonumber \\
    &=\sum_{i=1}^n{\left(E(X_i^4)-E(X_i^2)^2\right)} \nonumber\\
    &= \sum_{i=1}^n{\left(E(X_i^4) - (2)^2\right)}\label{eq:1}
\end{align}
In a symmetric distribution, all odd central moments are equal to zero. 
\begin{align*}
\mu_3  &= \sum_{k=0}^3 \left(\Comb{4}{k} \times E(X_i^k) \right)  & & \text{(by  \cref{eq:2})}\\
    &= E(X_i^3) - 4
\end{align*}
Equating $\mu_3$ to 0, we get,  $E(X_i^3) = 4.$\\
For a normal distribution, \begin{align}
    \text{kurtosis} = 4 
    &= \frac{\mu_4}{1^2} \nonumber\\
    &= \sum_{k=0}^4 \left(\Comb{4}{k} \times E(X_i^k) \right) & & \text{(by  \cref{eq:2})}\nonumber\\
     E(X_i^4) &= 10 \label{eq:3}
\end{align}
 From  \cref{eq:1} and \cref{eq:2},
 \begin{align*}
     Var(S_n) &= \sum_{i=1}^n{(10-4)}\\
     &= 6n\\
     \lim_{n \to \infty}{\frac{Var\brak{S_{n}}}{n}} &= 6
 \end{align*}
 Hence, option B is correct.
\end{document}

