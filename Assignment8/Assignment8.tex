\documentclass[journal,12pt,twocolumn]{IEEEtran}

\usepackage{setspace}
\usepackage{gensymb}
\singlespacing
\usepackage[cmex10]{amsmath}

\usepackage{amsthm}

\usepackage{mathrsfs}
\usepackage{txfonts}
\usepackage{stfloats}
\usepackage{bm}
\usepackage{cite}
\usepackage{cases}
\usepackage{subfig}

\usepackage{longtable}
\usepackage{multirow}

\usepackage{enumitem}
\usepackage{mathtools}
\usepackage{steinmetz}
\usepackage{tikz}
\usepackage{circuitikz}
\usepackage{verbatim}
\usepackage{tfrupee}
\usepackage[breaklinks=true]{hyperref}
\usepackage{graphicx}
\usepackage{tkz-euclide}
\usepackage{cleveref}
\usetikzlibrary{calc,math}
\usepackage{listings}
    \usepackage{color}                                            %%
    \usepackage{array}                                            %%
    \usepackage{longtable}                                        %%
    \usepackage{calc}                                             %%
    \usepackage{multirow}                                         %%
    \usepackage{hhline}                                           %%
    \usepackage{ifthen}                                           %%
    \usepackage{lscape}     
\usepackage{multicol}
\usepackage{chngcntr}

\DeclareMathOperator*{\Res}{Res}

\renewcommand\thesection{\arabic{section}}
\renewcommand\thesubsection{\thesection.\arabic{subsection}}
\renewcommand\thesubsubsection{\thesubsection.\arabic{subsubsection}}
\newcommand*{\Perm}[2]{{}^{#1}\!P_{#2}}%
\newcommand*{\Comb}[2]{{}^{#1}C_{#2}}%
\renewcommand\thesectiondis{\arabic{section}}
\renewcommand\thesubsectiondis{\thesectiondis.\arabic{subsection}}
\renewcommand\thesubsubsectiondis{\thesubsectiondis.\arabic{subsubsection}}
\renewcommand{\complement}[1]{\widetilde{#1}}   

\hyphenation{op-tical net-works semi-conduc-tor}
\def\inputGnumericTable{}                                 %%
\usepackage[utf8]{inputenc}
\usepackage[english]{babel}
\lstset{
%language=C,
frame=single, 
breaklines=true,
columns=fullflexible
}

\begin{document}
\newtheorem{theorem}{Theorem}[section]
\newtheorem{problem}{Problem}
\newtheorem{proposition}{Proposition}[section]
\newtheorem{prop}[theorem]{Proposition}
\newtheorem{lemma}{Lemma}[section]
\newtheorem{corollary}[theorem]{Corollary}
\newtheorem{example}{Example}[section]
%\newtheorem{proposition}{Proposition}[definition]
\theoremstyle{definition}
\newtheorem{definition}{Definition}[section]
\crefalias{prop}{Proposition} %
\newcommand{\BEQA}{\begin{eqnarray}}
\newcommand{\EEQA}{\end{eqnarray}}
\newcommand{\define}{\stackrel{\triangle}{=}}
\bibliographystyle{IEEEtran}
\raggedbottom
\setlength{\parindent}{0pt}
\providecommand{\mbf}{\mathbf}
\providecommand{\pr}[1]{\ensuremath{\pr\left(#1\right)}}
\providecommand{\qfunc}[1]{\ensuremath{Q\left(#1\right)}}
\providecommand{\sbrak}[1]{\ensuremath{{}\left[#1\right]}}
\providecommand{\lsbrak}[1]{\ensuremath{{}\left[#1\right.}}
\providecommand{\rsbrak}[1]{\ensuremath{{}\left.#1\right]}}
\providecommand{\brak}[1]{\ensuremath{\left(#1\right)}}
\providecommand{\lbrak}[1]{\ensuremath{\left(#1\right.}}
\providecommand{\cbrak}[1]{\ensuremath{\left\{#1\right\}}}
\providecommand{\lcbrak}[1]{\ensuremath{\left\{#1\right.}}
\providecommand{\rcbrak}[1]{\ensuremath{\left.#1\right\}}}
\theoremstyle{remark}
\newtheorem{rem}{Remark}
\newcommand{\sgn}{\mathop{\mathrm{sgn}}}
% \providecommand{\abs}[1]{\left\vert#1\right\vert}
% \providecommand{\res}[1]{\Res\displaylimits_{#1}} 
% \providecommand{\norm}[1]{\left\lVert#1\right\rVert}

% \providecommand{\mtx}[1]{\mathbf{#1}}
% \providecommand{\mean}[1]{E\left[ #1 \right]}
% \providecommand{\fourier}{\overset{\mathcal{F}}{ \rightleftharpoons}}

\providecommand{\system}{\overset{\mathcal{H}}{ \longleftrightarrow}}

\newcommand{\solution}{\noindent \textbf{Solution: }}
\newcommand{\cosec}{\,\text{cosec}\,}
\providecommand{\dec}[2]{\ensuremath{\overset{#1}{\underset{#2}{\gtrless}}}}
\newcommand{\myvec}[1]{\ensuremath{\begin{pmatrix}#1\end{pmatrix}}}
\newcommand{\mydet}[1]{\ensuremath{\begin{vmatrix}#1\end{vmatrix}}}
\numberwithin{equation}{subsection}
\makeatletter
\@addtoreset{figure}{problem}
\makeatother
\let\StandardTheFigure\thefigure
\let\vec\mathbf
\renewcommand{\thefigure}{\theproblem}
\def\putbox#1#2#3{\makebox[0in][l]{\makebox[#1][l]{}\raisebox{\baselineskip}[0in][0in]{\raisebox{#2}[0in][0in]{#3}}}}
     \def\rightbox#1{\makebox[0in][r]{#1}}
     \def\centbox#1{\makebox[0in]{#1}}
     \def\topbox#1{\raisebox{-\baselineskip}[0in][0in]{#1}}
     \def\midbox#1{\raisebox{-0.5\baselineskip}[0in][0in]{#1}}
\vspace{3cm}
\title{AI1103: Assignment 8}
\author{Chitneedi Geetha Sowmya \\ CS20BTECH11011}
\maketitle
\newpage
\bigskip
\renewcommand{\thefigure}{\theenumi}
\renewcommand{\thetable}{\theenumi}


Download all latex codes from 
\begin{lstlisting}
https://github.com/Geetha495/Assignment8/blob/main/Assignment8.tex
\end{lstlisting}
% Download all python codes from 
% \begin{lstlisting}
% https://github.com/Geetha495/Assignment8/blob/main/Assignment8.py
% \end{lstlisting}
\section{Problem}
Let $\phi(t)$ be a characteristic function of some random variable.  Then, which of the following is also a characteristic function ?
\begin{enumerate}
    \item $f(t) = [\phi(t)]^2$ for all $t \in \mathbb{R}$
    \item $f(t) = |\phi(t)|^2$ for all $t \in \mathbb{R}$
    \item $f(t) = \phi(-t)$ for all $t \in \mathbb{R}$
    \item $f(t) = \phi(t+1)$ for all $t \in \mathbb{R}$
\end{enumerate}
\section{Solution}
\begin{definition}[Characteristic Function]
 The function $\phi_X(t) = E(e^{itX})$ is called the characteristic function (cf ) of random variable $X$.
 \label{def:characterstic_function}
\end{definition}

\begin{proposition}[Properties of a Characteristic function]
All cf’s have the following properties:\\
\begin{enumerate}
    \item $\phi(-t) =\overline{\phi(t)}$ (complex conjugate)
    \item  The characteristic function of  $ -X$ is the complex conjugate $\overline{\phi(t)}$.
\end{enumerate}
\label{prop:properties_of_cf}
\end{proposition}
\begin{proposition}[Cf of sum of independent r.v.’s]
 If $X$ and $Y$ are independent, then
 \begin{align*}
     \phi_{X+Y}(t) = \phi_X(t)\times\phi_Y(t)
 \end{align*}
 \label{prop:Sum_of_independent_cvs}
\end{proposition}
Let $X$ be the given random variable and let $Y$  and $-X$ have the same distribution.
\begin{enumerate}
    \item \begin{align*} [\phi_X(t)]^2 &= \phi_X(t)\times\phi_X(t) \\
                &= \phi_{2X}(t) & & \text{(by  \cref{prop:Sum_of_independent_cvs})}
\end{align*}
Thus, $f(t) = [\phi(t)]^2$ is a characteristic function of random variable $2X$.

\item \begin{align*} |\phi_X(t)|^2 &= \phi_X(t)\times\overline{\phi_X(t)} \\
                &= \phi_X(t)\times\phi_Y(t) & & \text{(by  \cref{prop:properties_of_cf})}\\
                &= \phi_{X+Y}(t)
\end{align*}
Thus, $f(t)=|\phi(t)|^2$ is a characteristic function of random variable $(X+Y)$.\\

\item \begin{align*} \phi_X(-t) &= E(e^{i(-t)X}) & & \text{(by  \cref{def:characterstic_function})} \\
                &= E(e^{it(-X)}) \\
                &= E(e^{itY})\\
                &= \phi_Y(t)
\end{align*}
Thus, $f(t) = \phi(-t)$ is a characteristic function of random variable $Y$.

\item \begin{align*} \phi_X(t+1) &= E(e^{i(t+1)X})  & & \text{(by  \cref{def:characterstic_function})} \\
&= E(e^{itX}\times e^{iX}) 
\end{align*}
Thus, $f(t) = \phi(t+1)$ is a not a characteristic function.\\
\end{enumerate}


Hence, correct options are 1, 2, 3.


\end{document}



